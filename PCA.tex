\subsection{Principal Component Analysis}
\label{sub:PCA}
Principal Component Analysis (PCA) is a statistical technique that can be used to identify patterns in high dimensional data, for example images.
Examples in this section are inspired by the PCA tutorial by Smith \cite{smith2002tutorial}.

\subsubsection{How To Use}
\paragraph{Input data formatting.}
First of all, the data should be prepared for PCA.
In the case of images, the image should be flattened to a single vector.
This means an image of size $M$ by $N$ with $Z$ dimensions (colour values) should turn into a vector with length $M\times N\times Z$.
Turning the images into grayscale are often a good idea as they reduce the complexity by a third, and the added colour information may not be valuable.
A mean vector of all the input vectors should then be calculated and subtracted from the input vectors, producing a new input set with a mean of 0.
For 2 grayscale images of size $2\times 2$ pixels, the process could look like table \ref{tab:formattingexample}.
All of the adjusted input vectors should then be placed in a matrix, where every row is an adjusted image vector.

\begin{table}
\centering
\begin{tabular}{l|rrrr}
\hline
\noalign{\smallskip}
Vector & $(X_1, Y_1)$ & $(X_2, Y_1)$ & $(X_1, Y_2)$ & $(X_2, Y_2)$\\
\noalign{\smallskip}
\hline
\noalign{\smallskip}
$Image_1$ & 150 & 220 & 123 & 136 \\
$Image_2$ & 20 & 110 & 240 & 11 \\
Mean & 85 & 165 & 181.5 & 73.5 \\
$AdjustedImg_1$ & 65  & 55 & -58.5 & 62.5 \\
$AdjustedImg_2$ & -65 & -55 & 58.5 & -62.5 \\
AdjustedMean & 0 & 0 & 0 & 0 \\
\hline
\end{tabular}
\caption{Input data formatting example}\label{tab:formattingexample}
\end{table}

\paragraph{Calculate the covariance matrix.}
The covariance between two dimensions can be defined as follows, assuming the mean has already been subtracted from $X$ and $Y$.
$$cov(X, Y) = \frac{\sum_{i=1}^{n} X_iY_i}{n-1}$$
The covariance should be calculated for all dimensions.
For example, for a 3 dimensional data set with dimensions $(x,y,z)$ the covariance can be calculated for $(x,y)$, $(x,z)$ and $(y,z)$.
The covariance matrix for a data set with n dimensions is defined as follows.
$$C^{n\times n} = (c_{i,j}, c_{i,j} = cov(Dim_i, Dim_j))$$
The covariance matrix for the previous imaginary data set with dimensions $(x,y,z)$ is the following.
$$
C= \begin{pmatrix}
cov(x,x) & cov(x,y) & cov(x,z) \\
cov(y,x) & cov(y,y) & cov(y,z) \\
cov(z,x) & cov(z,y) & cov(z,z) \\
\end{pmatrix}
$$
The matrix is a square $n\times n$ matrix, and is symmetrical around the main diagonal, as $cov(a,b) = cov(b,a)$.
It is also worth noting that down the main diagonal the value is the covariance between a dimension and itself.

\paragraph{Calculate eigenvectors and eigenvalues of the covariance matrix.}
Eigenvectors are vectors that when multiplied with another matrix work like a constant. For example:
$$
\begin{pmatrix}
2 & 3 \\
2 & 1
\end{pmatrix}
\times
\begin{pmatrix}
3\\
2
\end{pmatrix}
=
\begin{pmatrix}
12 \\
8
\end{pmatrix}
=
4 \times
\begin{pmatrix}
3 \\
2
\end{pmatrix}
$$
Some properties of eigenvectors: eigenvectors of a matrix can only be found for square matrices, and not every square matrix has eigenvectors. 
Given an $n\times n$ matrix that does have eigenvectors, there are $n$ of them.
There is no easy way to calculate eigenvectors, but most programming languages have libraries with support for calculating them.
%Skriv eventuelt mere om eigenvektorer og værdier..?

Eigenvalues are closely related to eigenvector and there was one in the previous example, namely 4.
Eigenvalues comes in pairs with eigenvectors.

\paragraph{Choosing components.}
The eigenvector with the highest eigenvalue is the principle component of the data set.
Sorting the eigenvectors by eigenvalue allows us to see what components are most important, and allows us to ignore insignificant components with low eigenvalues if necessary.
After eliminating insignificant principal components, we can then form the feature vector which is a matrix of the eigenvectors we want to keep.


\paragraph{Deriving the new data set.}
The final step of Principal Component analysis is to apply our feature vector to the adjusted input data where the mean has been subtracted.
The input data is transposed to get the data items in the columns and the dimensions in the rows.
$$FinalData = FeatureVector\times AdjustedInputData^\top $$ %\top for transpose
This gives us the original data expressed in terms of the patterns identified.
%Mangler at skrive om PCA i forhold til vores project.

%\paragraph{Recovering old data}

\subsubsection{Generative model}
PCA is also a generative model.
With a means-image, you can multiple the principal components with various weights to see what effect the principal component has.
