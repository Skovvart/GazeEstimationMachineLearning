\section{Conclusion}
In this project we have finished an on-line Caltech course on Machine Learning and learned python.
We have successfully evaluated how Principal Component Analysis performs when doing classification-based gaze estimation.
Training data led us to initially believe that the data was completely separable, with up to 90\% accurate identification of all 4 points, but this turned out to not be applicable on the testing data set.
We discovered that we were only able to perform left-right separation on both infra-red and regular images with more than 80\% accuracy, while identification of all points could only be done with about 40\% accuracy in both cases.

Future work should attempt to further improve the accuracy, possibly by normalizing the eye images further.
These experiments could  be performed fairly easily if the required manual labour could be reduced, mainly when it comes to marking the eye-corners.
Eye-identification algorithms could come in handy here, but accuracy is important.