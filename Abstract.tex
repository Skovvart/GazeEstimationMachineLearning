\begin{abstract}
% State the problem
%In this paper we attempt to classify which of 4 points an eye-image is looking at by using Principal Component Analysis.
% Say why it’s an interesting problem
%Gaze estimation has many applications, for example allowing the disabled to type with their eyes, and using machine learning to perform gaze estimation could reduce its cost.
% Say what your solution achieves
% Say what follows from your solution

In this paper we attempt to perform Principal Component Analysis on a set of eye-images and evaluate the feasibility of performing discrete gaze estimation using this approach.
Gaze estimation has many applications, for example allowing the disabled to type using their eyes. Using machine learning could reduce the cost of performing gaze estimation.
The data used for learning is gathered under an experiment where test-subjects are looking at 4 points placed in a straight line marked with both infra-red- and coloured lights.
Our results show that it is possible to separate right from left in more than 80\% of cases, but correctly separating all 4 points can only be done in about 40\% of the cases.
Future work can hopefully increase the accuracy in both cases.

To perform this experiment we have taken an on-line machine learning course from Caltech and have taught ourself Python.
The source code for this project can be found on Github (\url{https://www.dropbox.com/sh/is6qmdl2w6costb/5GwYB8chle}).
\\\\
\textbf{Keywords:} Principal Component Analysis, Discrete Gaze Estimation, Machine Learning
\end{abstract}
