\section{Threats To Validity}
\label{sec:Threats}
\subsection{Internal Threats}
\paragraph{Human error during classification.}
The corners of the eye and the centre of the pupil were manually marked and with less than 100\% accuracy.
This introduces the possibility that some of the images were wrongfully extracted and this may have impacted learning.
It would be nice if it could be automated using image analysis to extract the features of the eyes, but that also comes with the risk of being done poorly, and identifying the bad results would most likely have to be done manually in any case.

\paragraph{Web-cam quality.}
The quality of the web-cam images were not very high.
This makes the potential usability of our solution cheaper which is a plus, but it may impact learning results.
This does not seem to be a significant factor, as our tests showed that not much information was lost when the images were rescaled.

\paragraph{Test set-up.}
Our test set-up required the test-subjects to sit a certain distance from the camera for the infra-red lights to be visible as glints in the eye.
This had the effect that the eye was a smaller part of the input images than they could have been.

The test set-up was also somewhat loose, and heads moved a little during head-still sequences.
Some noise like this is to be expected, but it may have had a small effect on learning.

\paragraph{Resizing of learning data.}
Images were scaled down to $20\times 20$ pixels to simplify the learning process and this could potentially have an impact on learning.
From our results, the rescaling only had a minor effects.

\paragraph{Space surrounding the eyes.}
The area surrounding the eyes vary a lot, both in location compared to the eye and in size. 
It may be beneficial to stretch the eye to fill the entire eye-image to reduce the impact of this.

\subsection{External Threats}
\paragraph{General applicability.}
Eyes come in many shapes and forms and can vary greatly in shape, scale and location.
We only tested on a few demographics and may have gotten different results for others.

Given proper test set-up, it is to be expected that you could use our approach to learn what points that demographic is looking at, but the same training data set may not work equally well for all demographics.
